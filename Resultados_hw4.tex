\documentclass{article}
\usepackage[utf8]{inputenc}
\usepackage[letterpaper, portrait, margin=0.9in]{geometry}
\usepackage{amsmath}
\usepackage[spanish,es-nodecimaldot]{babel}
\usepackage{subfig}
\usepackage{graphicx}

\title{Tarea 4 - Métodos computacionales}
\author{Santiago Valencia - 201532432}
\date{19 de noviembre 2018}


\begin{document}

\maketitle

\section*{Ecuaciones diferenciales ordinarias}

La solución de la ecuación diferencial:

\begin{equation*}
    \frac{d^2\vec{x}(t)}{dt^2} = -\vec{g} - c\frac{|\vec{v}(t)|^2}{m}\frac{\vec{v}(t)}{|\vec{v}(t)|}
\end{equation*}
se puede encontrar si se descompone la ecuación vectorial en sus coordenadas $(x, y)$ y además se sustituye $\vec{v} = d\vec{x}/dt$:
\begin{gather*}
    \frac{dv_x}{dt} = -c\frac{|\vec{v}|}{m}v_x \ ; \ \frac{dx}{dt} = v_x \\
    \frac{dv_y}{dt} = -g - c\frac{|\vec{v}|}{m}v_y \ ; \ \frac{dy}{dt} = v_y
\end{gather*}

A partir de lo anterior se aplica el método Runge-Kutta de cuarto orden para los pares $x, v_x$ y $y, v_y$ para obtener las siguientes gráficas ($c = 0.2, g = 10 \text{ m/s}^2, m = 0.2 \text{ kg}, v(0) = 300 \text{ m/s}, (x_0, y_0) = (0, 0)$):

\begin{figure}[h]
\centering
    \subfloat[Trayectoria para 45 grados. ]{\includegraphics[width = 0.45\linewidth]{Tray_45.pdf}}
    \subfloat[Trayectorias para varios ángulos de lanzamiento.]{\includegraphics[width=0.45\linewidth]{Tray_varias.pdf}}
    \label{fig:aux}
    \caption{Gráficas de la solución de la ecuación diferencial ordinaria.}
\end{figure}

Gracias a las gráficas se puede ver que, contrario a lo que sucede en el caso sin fricción, el ángulo entre los ensayados que produce el mayor rango es el de 20 grados.




\section*{Ecuaciones diferenciales parciales}

\end{document}
